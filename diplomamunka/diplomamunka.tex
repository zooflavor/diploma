\documentclass[12pt]{report}
\usepackage[a4paper,
			inner = 35mm,
			outer = 25mm,
			top = 25mm,
			bottom = 25mm]{geometry}
\usepackage{lmodern}
\usepackage[magyar]{babel}
\usepackage[utf8]{inputenc}
\usepackage[T1]{fontenc}
\usepackage[unicode]{hyperref}
\usepackage{graphicx}
\usepackage{amssymb}
\usepackage{amsmath}
\usepackage{epstopdf}
\usepackage{setspace}
\usepackage[nottoc,numbib]{tocbibind}
\usepackage{color}
\setcounter{secnumdepth}{3}
\usepackage[chapter]{algorithm}
\usepackage{algorithm}
\usepackage{algorithmicx}
\usepackage{algpseudocode}
\usepackage{cellspace}
\usepackage{float}
\usepackage{listings}
\usepackage{moresize}
\usepackage{multirow}
\usepackage{pgfplots}
%\usepackage{siunitx}
\usepackage{tikz}
\usepackage{tikz-qtree}
\usepackage{titlesec}
\pgfplotsset{compat=1.9}
\onehalfspacing
\lstset{basicstyle=\ssmall\tt}
\lstset{literate=
  {á}{{\'a}}1 {é}{{\'e}}1 {í}{{\'i}}1 {ó}{{\'o}}1 {ú}{{\'u}}1
  {Á}{{\'A}}1 {É}{{\'E}}1 {Í}{{\'I}}1 {Ó}{{\'O}}1 {Ú}{{\'U}}1
  {à}{{\`a}}1 {è}{{\`e}}1 {ì}{{\`i}}1 {ò}{{\`o}}1 {ù}{{\`u}}1
  {À}{{\`A}}1 {È}{{\'E}}1 {Ì}{{\`I}}1 {Ò}{{\`O}}1 {Ù}{{\`U}}1
  {ä}{{\"a}}1 {ë}{{\"e}}1 {ï}{{\"i}}1 {ö}{{\"o}}1 {ü}{{\"u}}1
  {Ä}{{\"A}}1 {Ë}{{\"E}}1 {Ï}{{\"I}}1 {Ö}{{\"O}}1 {Ü}{{\"U}}1
  {â}{{\^a}}1 {ê}{{\^e}}1 {î}{{\^i}}1 {ô}{{\^o}}1 {û}{{\^u}}1
  {Â}{{\^A}}1 {Ê}{{\^E}}1 {Î}{{\^I}}1 {Ô}{{\^O}}1 {Û}{{\^U}}1
  {œ}{{\oe}}1 {Œ}{{\OE}}1 {æ}{{\ae}}1 {Æ}{{\AE}}1 {ß}{{\ss}}1
  {ű}{{\H{u}}}1 {Ű}{{\H{U}}}1 {ő}{{\H{o}}}1 {Ő}{{\H{O}}}1
  {ç}{{\c c}}1 {Ç}{{\c C}}1 {ø}{{\o}}1 {å}{{\r a}}1 {Å}{{\r A}}1
  {€}{{\euro}}1 {£}{{\pounds}}1 {«}{{\guillemotleft}}1
  {»}{{\guillemotright}}1 {ñ}{{\~n}}1 {Ñ}{{\~N}}1 {¿}{{?`}}1
}
\babelhyphenation[magyar]{Tanszék}
\babelhyphenation[magyar]{pszeudo-prím}
\babelhyphenation[magyar]{prím-eket}
\begin{document}
\begin{titlepage}
\vspace*{0cm}
\centering
\begin{tabular}{cp{2cm}c}
\begin{minipage}{4cm}
\vspace{0pt}
\includegraphics[width=1\textwidth]{eltecimerszines}
\end{minipage} & &
\begin{minipage}{7cm}
\vspace{0pt}Eötvös Loránd Tudományegyetem \vspace{10pt} \newline
Informatikai Kar \vspace{10pt} \newline
Algoritmusok és Alkalmazásaik \newline
Tanszék
\end{minipage}
\end{tabular}

\vspace*{0.2cm}
\rule{\textwidth}{1pt}

\vspace*{6cm}
{\Huge Cache-optimális algoritmusok elemzése}

\vspace*{5cm}
\begin{tabular}{lp{3cm}l}
Szabó László Ferenc & & Nagy Péter\\
Docens & & Programtervező Informatikus MSc
\end{tabular}

\vfill

\vspace*{1cm}
Budapest, 2023
\end{titlepage}

\tableofcontents

\chapter{Bevezetés}

Yada-yada, ma már kenyérpirítókba is olyan számítógépet tesznek, aminek több memóriaszintje van. Valamint kíváncsi vagyok, hogy a szakdolgozathoz megálmodott szita mennyire cache-oblivious.

\chapter{Cache modellje}

\begin{itemize}
\item external memory model
\item két féle műveletigény: lépések száma vs memóriaműveletek száma
\item cache aware algoritmus
\item cache replacement policy, MIN, LRU, FIFO
\item cache oblivious algoritmus
\item cache miss okai
\end{itemize}

\section{MIN}

MIN avagy OPT policy-t ismerjük off-line esetben. MIN és LRU is rendelkezik azzal a tulajdonsággal, hogy egy nagyobb cache minden elemet tartalmaz, amit egy kisebb cache tartalmazna, egy adott műveletsorozaton.

\section{LRU lemma}

Nem túl szigorú feltételek mellett LRU aszimptotikusan ugyanolyan jó, mint a MIN. Elemzésnél választhatunk, hogy melyiket használjuk.

\chapter{Cache oblivious agoritmusok}

\section{Lista}

Ha a memóriában nem túl sok egybefűggő részen vannak sorban a lista elemei, akkor az cache-oblivious végigolvasni.

\section{Rendezés}

Oszd meg és uralkodj. 2 algoritmus is van. Az alsó korlát elérhető, de a papír nem az igazi erről. Tall cache assumption. A tall cache assumption lényegesen befolyásolja a műveletigényt.

\section{Mátrix szorzás}

Oszd meg és uralkodj.

\section{FFT}

Oszd meg és uralkodj. Alsó korlát is van, amit elér, de túl erős feltevésekkel.

\section{Diff.egyenlet közelítés}

Oszd meg és uralkodj. Több dimenzióban.

\section{Statikus keresőfa}

Van Emde Boas fák. Dinamikus eset szuperbonyolult, asszem.

\section{Edénysor szita}

\subsection{A feladat, a fa, és a program}

Legyen $a, n \in \mathbb{N}$, $a \ge 2$.
Szeretnénk meghatározni az $a^n$-nél kisebb prímszámokat.

A feladat megoldásához Eratoszthenész szitáját alkalmazzuk fákra, a főbb tulajdonságok megőrzésével.
\begin{itemize}
\item Egy szám prímtulajdonságát minden prímosztójának meghatározásával döntjük el.
\item Egy prímszám többszöröseit más prímszámoktól függetlenül vizsgáljuk.
\item A számokról növekvő sorrendben döntjük el a prímtulajdonságot.
\end{itemize}

A műveletigény becsléséhez egy egyszerűbb feladatot választunk.

Legyen $a$ a helyi értékes számrendszer alapja.
Legyen $A = \lbrace 0, 1, \dots, a-1 \rbrace$ az ábécé.
Tekintsük az $n$ hosszú szavak kódfáját.
Legyen ez a fa $F_{a,n}$.
Minden $p \in \mathbb{N}$-re $F_{a,n,p}$ legyen a gyökércsúcs és a $p$ többszöröseinek megfelelő levelek által indukált részfa.
Adjunk algoritmust, ami minden $p < a^n$ prímszámra meglátogatja $F_{a,n,p}$ összes csúcsát, és megjelöli ezeket a csúcsokat $p$-vel.

\begin{figure}[h]
\centering
\begin{tikzpicture}[level distance=50pt, sibling distance=5pt]
\Tree
[.\node[draw,dashed,red]{$0-15$}; 
	\edge[dashed,red] node[auto=right] {$0$};
	[.\node[draw,dashed,red]{$0-7$};
		\edge[dashed,red] node[auto=right] {$0$};
		[.\node[draw,dashed,red]{$0-3$};
			\edge[dashed,red] node[auto=right] {$0$};
			[.\node[draw,dashed,red]{$0-1$};
				\edge[dashed,red] node[auto=right] {$0$};
				[.\node[draw,dashed,red]{$0$}; ]
				\edge[] node[auto=left] {$1$};
				[.\node[draw]{$1$}; ]
			]
			\edge[] node[auto=left] {$1$};
			[.\node[draw]{$2-3$};
				\edge[] node[auto=right] {$0$};
				[.\node[draw]{$2$}; ]
				\edge[] node[auto=left] {$1$};
				[.\node[draw]{$3$}; ]
			]
		]
		\edge[dashed,red] node[auto=left] {$1$};
		[.\node[draw,dashed,red]{$4-7$};
			\edge[dashed,red] node[auto=right] {$0$};
			[.\node[draw,dashed,red]{$4-5$};
				\edge[] node[auto=right] {$0$};
				[.\node[draw]{$4$}; ]
				\edge[dashed,red] node[auto=left] {$1$};
				[.\node[draw,dashed,red]{$5$}; ]
			]
			\edge[] node[auto=left] {$1$};
			[.\node[draw]{$6-7$};
				\edge[] node[auto=right] {$0$};
				[.\node[draw]{$6$}; ]
				\edge[] node[auto=left] {$1$};
				[.\node[draw]{$7$}; ]
			]
		]
	]
	\edge[dashed,red] node[auto=left] {$1$};
	[.\node[draw,dashed,red]{$8-15$};
		\edge[dashed,red] node[auto=right] {$0$};
		[.\node[draw,dashed,red]{$8-11$};
			\edge[] node[auto=right] {$0$};
			[.\node[draw]{$8-9$};
				\edge[] node[auto=right] {$0$};
				[.\node[draw]{$8$}; ]
				\edge[] node[auto=left] {$1$};
				[.\node[draw]{$9$}; ]
			]
			\edge[dashed,red] node[auto=left] {$1$};
			[.\node[draw,dashed,red]{$10-11$};
				\edge[dashed,red] node[auto=right] {$0$};
				[.\node[draw,dashed,red]{$10$}; ]
				\edge[] node[auto=left] {$1$};
				[.\node[draw]{$11$}; ]
			]
		]
		\edge[dashed,red] node[auto=left] {$1$};
		[.\node[draw,dashed,red]{$12-15$};
			\edge[] node[auto=right] {$0$};
			[.\node[draw]{$12-13$};
				\edge[] node[auto=right] {$0$};
				[.\node[draw]{$12$}; ]
				\edge[] node[auto=left] {$1$};
				[.\node[draw]{$13$}; ]
			]
			\edge[dashed,red] node[auto=left] {$1$};
			[.\node[draw,dashed,red]{$14-15$};
				\edge[] node[auto=right] {$0$};
				[.\node[draw]{$14$}; ]
				\edge[dashed,red] node[auto=left] {$1$};
				[.\node[draw,dashed,red]{$15$}; ]
			]
		]
	]
]
\end{tikzpicture}
\caption{$F_{2,4}$ és \color{red} $F_{2,4,5}$}
\label{fig:F25}
\end{figure}

Tegyük fel, hogy ismerjük az $a^n$-nél kisebb prímszámokat, és legyen ezek halmaza $P$.
Ekkor a \ref{alg:tree-factors} algoritmussal $F_{a,n}$ minden $u$ csúcsát meg tudjuk jelölni minden olyan $p$ prímszámmal, hogy az $u$ csúcs $F_{a,n,p}$ csúcsa is.

\begin{algorithm}
\floatname{algorithm}{Algoritmus}
\caption{Az $[u, v)$ intervallum szitálása}
\label{alg:tree-factors}
\begin{algorithmic}[1]
\Require $a$: számrendszer alapja
\Require $n$: kódfa magassága
\Require $u$: $F_{a,n}$ egy csúcsát azonosító szó
\Require $P$: azok a $p$ prímszámok, hogy létezik $p$-nek olyan $v$ többszöröse, hogy $u$ prefixe $v$-nek
\Procedure{FaBejárás}{$a$, $n$, $u$, $P$}
	\State legyenek a számok az $[u, v)$ intervallumban megjelöletlenek
	\For{$p \in \{\textrm{prímek}\sqrt{v-1}\textrm{-ig}\}$}
		\For{$f \gets \max \lbrace p^2, p \left \lceil{\frac{u}{p}}\right \rceil \rbrace ; v > f; f \gets f+p$}
			\State legyen $f$ megjelölve
		\EndFor
	\EndFor
	\For{$f \in [u, v)$}
		\If{$f$ nincs megjelölve}
			\State $f$ prím
		\EndIf
	\EndFor
\EndProcedure
\end{algorithmic}
\end{algorithm}

\subsection{A program műveletigénye a fában}

\subsection{A program műveletigénye gyorsítótárral a fában}

\subsection{Feltételek a gyorsítótár méretére}

\subsection{Tetszőleges n-ig, nem csak teljes fára}

\subsection{Az igazi program fix méretű számokkal}

\subsection{Az igazi program változó méretű számokkal}

\chapter{Szimulátor}

\section{Igazi cache}

Asszociativitás. Több szint. Több mag osztozik. Léteznek egészen speciális cache-ek, TLB, decoded microcode, trace cache, ...

\section{Megoldás}

\section{Választható cache-ek}

\begin{itemize}
\item utasítás/adat/mindkettő
\item MIN
\item LRU
\item FIFO, LIFO, LFU, ..., kevésbé izgalmasak
\item teljes/1-2-3...-n asszociativitás
\item mindenféle méretben
\item egymás után akár több is
\end{itemize}

\begin{thebibliography}{9}

\bibitem{atkin}
A. O. L. Atkin, D. J. Bernstein: Prime sieves using binary quadratic forms, Mathematics of Computation, Volume 73 (2004) 1023–1030

\end{thebibliography}

\end{document}

%%% Local Variables:
%%% mode: latex
%%% TeX-master: t
%%% End:
