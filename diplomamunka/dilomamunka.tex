\documentclass[12pt]{report}
\usepackage[a4paper,
			inner = 35mm,
			outer = 25mm,
			top = 25mm,
			bottom = 25mm]{geometry}
\usepackage{lmodern}
\usepackage[magyar]{babel}
\usepackage[utf8]{inputenc}
\usepackage[T1]{fontenc}
\usepackage[unicode]{hyperref}
\usepackage{graphicx}
\usepackage{amssymb}
\usepackage{amsmath}
\usepackage{epstopdf}
\usepackage{setspace}
\usepackage[nottoc,numbib]{tocbibind}
\usepackage{color}
\setcounter{secnumdepth}{3}
\usepackage[chapter]{algorithm}
\usepackage{algorithm}
\usepackage{algorithmicx}
\usepackage{algpseudocode}
\usepackage{cellspace}
\usepackage{float}
\usepackage{listings}
\usepackage{moresize}
\usepackage{multirow}
\usepackage{pgfplots}
\usepackage{siunitx}
\usepackage{tikz}
\usepackage{titlesec}
\pgfplotsset{compat=1.9}
\onehalfspacing
\lstset{basicstyle=\ssmall\tt}
\lstset{literate=
  {á}{{\'a}}1 {é}{{\'e}}1 {í}{{\'i}}1 {ó}{{\'o}}1 {ú}{{\'u}}1
  {Á}{{\'A}}1 {É}{{\'E}}1 {Í}{{\'I}}1 {Ó}{{\'O}}1 {Ú}{{\'U}}1
  {à}{{\`a}}1 {è}{{\`e}}1 {ì}{{\`i}}1 {ò}{{\`o}}1 {ù}{{\`u}}1
  {À}{{\`A}}1 {È}{{\'E}}1 {Ì}{{\`I}}1 {Ò}{{\`O}}1 {Ù}{{\`U}}1
  {ä}{{\"a}}1 {ë}{{\"e}}1 {ï}{{\"i}}1 {ö}{{\"o}}1 {ü}{{\"u}}1
  {Ä}{{\"A}}1 {Ë}{{\"E}}1 {Ï}{{\"I}}1 {Ö}{{\"O}}1 {Ü}{{\"U}}1
  {â}{{\^a}}1 {ê}{{\^e}}1 {î}{{\^i}}1 {ô}{{\^o}}1 {û}{{\^u}}1
  {Â}{{\^A}}1 {Ê}{{\^E}}1 {Î}{{\^I}}1 {Ô}{{\^O}}1 {Û}{{\^U}}1
  {œ}{{\oe}}1 {Œ}{{\OE}}1 {æ}{{\ae}}1 {Æ}{{\AE}}1 {ß}{{\ss}}1
  {ű}{{\H{u}}}1 {Ű}{{\H{U}}}1 {ő}{{\H{o}}}1 {Ő}{{\H{O}}}1
  {ç}{{\c c}}1 {Ç}{{\c C}}1 {ø}{{\o}}1 {å}{{\r a}}1 {Å}{{\r A}}1
  {€}{{\euro}}1 {£}{{\pounds}}1 {«}{{\guillemotleft}}1
  {»}{{\guillemotright}}1 {ñ}{{\~n}}1 {Ñ}{{\~N}}1 {¿}{{?`}}1
}
\babelhyphenation[magyar]{pszeudo-prím}
\babelhyphenation[magyar]{prím-eket}
\begin{document}
\begin{titlepage}
\vspace*{0cm}
\centering
\begin{tabular}{cp{2cm}c}
\begin{minipage}{4cm}
\vspace{0pt}
\includegraphics[width=1\textwidth]{eltecimerszines}
\end{minipage} & &
\begin{minipage}{7cm}
\vspace{0pt}Eötvös Loránd Tudományegyetem \vspace{10pt} \newline
Informatikai Kar \vspace{10pt} \newline
Algoritmusok és Alkalmazásaik Tanszék
\end{minipage}
\end{tabular}

\vspace*{0.2cm}
\rule{\textwidth}{1pt}

\vspace*{6cm}
{\Huge Cache-optimális algoritmusok elemzése}

\vspace*{5cm}
\begin{tabular}{lp{3cm}l}
Szabó László Ferenc & & Nagy Péter\\
Habilitált egyetemi docens & & Programtervező Informatikus MSc
\end{tabular}

\vfill

\vspace*{1cm}
Budapest, 2022
\end{titlepage}

\tableofcontents

\chapter{Bevezetés}

Yada-yada, ma már kenyérpirítókba is olyan számítógépet tesznek, aminek több memóriaszintje van. Valamint kíváncsi vagyok, hogy a szakdolgozathoz megálmodott szita mennyire cache-oblivious.

\chapter{Cache modellje}

\begin{itemize}
\item external memory model
\item két féle műveletigény: lépések száma vs memóriaműveletek száma
\item cache aware algoritmus
\item cache replacement policy, MIN, LRU, FIFO
\item cache oblivious algoritmus
\item cache miss okai
\end{itemize}

\section{MIN}

MIN avagy OPT policy-t ismerjük off-line esetben. MIN és LRU is rendelkezik azzal a tulajdonsággal, hogy egy nagyobb cache minden elemet tartalmaz, amit egy kisebb cache tartalmazna, egy adott műveletsorozaton.

\section{LRU lemma}

Nem túl szigorú feltételek mellett LRU aszimptotikusan ugyanolyan jó, mint a MIN. Elemzésnél választhatunk, hogy melyiket használjuk.

\chapter{Cache oblivious agoritmusok}

\section{Lista}

Ha a memóriában nem túl sok egybefűggő részen vannak sorban a lista elemei, akkor az cache-oblivious végigolvasni.

\section{Rendezés}

Oszd meg és uralkodj. 2 algoritmus is van. Az alsó korlát elérhető, de a papír nem az igazi erről. Tall cache assumption. A tall cache assumption lényegesen befolyásolja a műveletigényt.

\section{Mátrix szorzás}

Oszd meg és uralkodj.

\section{FFT}

Oszd meg és uralkodj. Alsó korlát is van, amit elér, de túl erős feltevésekkel.

\section{Diff.egyenlet közelítés}

Oszd meg és uralkodj. Több dimenzióban.

\section{Statikus keresőfa}

Van Emde Boas fák. Dinamikus eset szuperbonyolult, asszem.

\section{Szakdolgozat-szita, aminek nincs rendes neve}

A műveletigénynél kicsit majd erőszakoskodni kell tételekkel, amik a végtelenben igazak.

\section{Párhuzamos algoritmusok (nem lesz)}

Végül nem akarok velük foglalkozni. Van egy mendemonda, hogy cache oblivious algoritmusok aránylag jól tűrik, hogy más programokkal együtt futnak. Valamint a work stealing scheduler jól működik együtt a cache-sel, szintén csak mese.

\chapter{Szimulátor}

\section{Igazi cache}

Asszociativitás. Több szint. Több mag osztozik. Léteznek egészen speciális cache-ek, TLB, decoded microcode, trace cache, ...

\section{Megoldás}

C egy nagyon szűk része, csak annyi, hogy nagy int és float tömböket rekurzív függyvényekkel módosít. Nincs dinamikus allokáció. Nincs párhuzamosság. Nincs I/O. Nincs semmi op.rendszer.

Igazi gépen a kedvenc C fordítóval futtatható, és lehet méricskélni cache-grind-dal, vagy hardver számlálókkal.

Szimulálásnál valami saját belső gép utasításaira fordítja, ezt a gépet szimulálja, és minden memóriaműveletet naplóz. Utólag valamilyen cache-hierarchián meg lehet határozni a futási sebességet. (Ha csinálnék párhuzamos programokat, akkor nem lehet utólag sebességet rendelni a futáshoz, mert az utasítások sorrendjét befolyásolhatja a cache-re várakozás.)
A programot is memóriában kell tárolni.

\section{Választható cache-ek}

\begin{itemize}
\item utasítás/adat/mindkettő
\item MIN
\item LRU
\item FIFO, LIFO, LFU, ..., kevésbé izgalmasak
\item teljes/1-2-3...-n asszociativitás
\item mindenféle méretben
\item egymás után akár több is
\end{itemize}

\begin{thebibliography}{9}

\bibitem{atkin}
A. O. L. Atkin, D. J. Bernstein: Prime sieves using binary quadratic forms, Mathematics of Computation, Volume 73 (2004) 1023–1030

\end{thebibliography}

\end{document}

%%% Local Variables:
%%% mode: latex
%%% TeX-master: t
%%% End:
